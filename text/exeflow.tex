% exeflow.tex
% 9/5/2013 jichi

\section{Execution-flow modeling for MPI applications}
\label{sec:mpi:exeflow}

To analytically estimate the runtime for MPI applications,
a communication model is needed for projecting performance of MPI communication offline.
Based on the LogP theory,
We developed a communication model
to estimate the communication time for point-to-point or collective communication
given the message size and number of nodes.
The model could be formulated by the following equation:
\begin{equation}
runtime = F_{mpi\_type}(message\_size; node\_number)
\end{equation}
It takes the type of the MPI communication,
number of nodes, and data transfered,
and output the estimated communication time
excluding the balancing time.
The type of the MPI communication including point-to-point send/recv and collective functions.
In the framework, the communication model is created by profiling kernel MPI benchmarks over the target hardware.

The following subsections will discuss the communication model, the profiled kernel benchmarks,
and the statistical analysis for constructing the model.

\subsection{Communication cost and message size}
The communication cost can be estimated using LogGP model from
the process number ($p$),
message size ($n$),
and platform-specific constant parameters ($alpha$, $beta$, $gamma$, \ldots).
In the framework, MPICH is the MPI runtime environment on the target machines.
In MPICH 3.1,
the communication cost is usually proportional to the message size for both point-to-point and collective communication.

\subsubsection{MPI\_Send and MPI\_Recv}
According to the LogGP model, the point-to-point communication can be modeled as a linear equation for both small and large message sizes:
\begin{equation}
cost_{p2p} = alpha + n\cdot beta
\end{equation}

\subsubsection{MPI\_Alltoall}
In MPICH 3.1, the communication algorithm is different for small and large message sizes.
\begin{equation}
cost_{a2a}|_{n<threshold} = log p\cdot alpha + (n/2)\cdot log p\cdot beta
\end{equation}
\begin{equation}
cost_{a2a}|_{n>threshold} = (p-1)\cdot alpha + n\cdot beta
\end{equation}
In each case, the overall cost is proportional to the message size $n$.

\subsubsection{MPI\_Alltoallv}
In MPICH 3.1, the Alltoallv is implemented as decoupled non-blocking send and receives.
So, the communication cost should be proportional to the overall message size.

\subsubsection{MPI\_Bcast and MPI\_Reduce}
The cost for broadcast and reduce in MPICH 3.1 is as follows where the cost is proportional to the message size $n$.
\begin{equation}
cost_{bcast} = log p\cdot alpha + n\cdot log p\cdot beta
\end{equation}
\begin{equation}
cost_{reduce} = log p\cdot alpha + n\cdot log p\cdot beta + n\cdot log p\cdot gamma
\end{equation}

%  Cost = lgp.alpha + n.((p-1)/p).beta
%\subsubsection{MPI\_Gather}

\subsection{Analytic communication model}
Given the fact that the communication cost is proportional to the message size
but could the communication algorithm could change for small or large message size,
different MPI communication model can be generalized
as a broken-line linear equation:
\begin{equation}
cost(n;p) = \sum_i(alpha_{i,p} + n\cdot beta_{i,p})|_{threshold_i<n<threshold_{i+1}}
\end{equation}
Given fixed processor number $p$, the communication cost is a linear equation of
message size $n$ within a pair of message size thresholds.
The parameters $alpha$ and $beta$ are determined by the processor number and the range of the thresholds
and runtime-dependent on the target platform.

\subsection{Communication cost and communication time}
The communication cost and communication time are related, but not necessarily the same.
The communication time is not only effected by the communication cost,
but also include the wait time to balance different nodes.
For applications with unbalanced workloads,
the wait time could also become the performance bottlenecks.
Currently, the framework focuses only on
modeling balanced workload which is usually true for scientific applications,
and the wait time is ignored.
The communication time is approximated by the communication cost.
Modeling unbalanced workload is left for the future work.

\subsection{Kernel MPI applications}
A kernel MPI benchmark
is a small application that contains only one type MPI functions to model,
and report the average communication time over a large number of evaluations for the given message size and node number.
In the current framework,
  we wrote kernel benchmarks for send/recv, alltoall, alltoallv, broadcast and reduce.
  %which covers the communication bottlenecks in the NPB.
The kernel benchmarks are profiled on the target platform
to collect message sizes, number of nodes, and communication time.

\subsection{Linear regression}
Linear regression is used to calculate the linear parameters.
for the message size.
Given the experiment result,
  the message thresholds for the broken lines can be estimated
  from both the result message\_size-runtime figure
  and the platform-dependent MPI environment variables from \texttt{mpivars}.
Then, the linear parameters ($alpha$ and $beta$)
can be calculated by applying linear regression to the message sizes and runtime
for specific message thresholds, numbers of processor, and MPI communication type.

% EOF
