
\subsection{Safety Of Optimization}
\label{sec:cco:safety}

\todo {Make an algorithm Figure. Reference the algorithm figure, summarize the key steps:
Given the loops that containing the computation and communication to overlap,
  the safety analysis will check if it is safe to reorder them,
  and what variables are needed to be privatized to enable the overlap transformation.
}

\todo {don't make up terms such as CCO loop, it's not even a name}
\todo {reorganize the following to following an algorithm figure}

As shown in Figure~\ref{fig:ft} , loop can be split into separation of computation can communication.
After reordering the computation and communication for CCO,
  three loop iterations $I-1$, $I$ and $I+1$ will be \emph{interleaved}.
$After(I+1)$ will be executed before $Communication(I)$ and $Before(I-1)$.
To guarantee the safety of the reordering, there must be no loop-carried dependence from
  $After(I+1)$ to $Communication(I)$ and $Before(I-1)$.

However, in the applications such as NAS FT,
  there are often buffer arrays shared by different iterations that could introduce such loop-carried dependence.
When that happens,
  the buffers are needed to be privatized to each loop iteration in order to apply CCO optimization.
If the buffers are also used as the output of the loop,
  the privatized buffers are needed to be copied to the output array.
But if the dependence from $After(I-1)$ to $Before(I+1)$ cannot be removed by privatizing variables,
  it is not safe to apply CCO optimization to the loop.

After applying annotation-based inlining, function calls in the CCO loop will be eliminated.
Conventional loop dependence analysis can be applied report the variables that have the loop-carried dependence discussed in the first section.
Based on the location of the communication,
   all the statements in the CCO can be divided into three groups:
   (1) computation statements before the communication,
   (2) communication statements,
   and (3) computation statements after the communication.
Then, the variables that has loop-carried dependence from (3) to (1) and (2) can be identified
  using loop dependence hoisting analysis~\cite{YKA:JSC04}.

% EOF
