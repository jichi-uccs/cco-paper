% workflow.tex
% 9/5/2013
% The hotspot and hotpath algorithm figure.
%
% Algorithms:
% - region profiling
% - hot spot prediction
% - hot path extraction
%
%
% Example:
% - Input: FT
%   F1 // cffts
%     L
%       L
%        C1 // cfftz
%   F2 // transpose
%     F3 // transpose_local
%       L
%         L
%           C2
%     F4 // transpose_global
%       MPI_Alltoall
%
% - Inline & decompose computation:
%   L
%     L
%       C1
%   L
%     L
%       F3 // transpose_local
%   MPI_Alltoall
%
% - Output: one-sided:
%    MPI_Win_create
%    MPI_Win_fence
%    L
%      L
%        C1
%        F3
%        L
%          MPI_Put
%      MPI_Win_fence
%
% - Output: non-blocking, see: http://www.2decomp.org/occ.html
%   L
%     C1
%     F3 // transpose_local
%   MPI_Isend D_1
%   MPI_Irecv D_1
%   L k=2,n
%     L
%       C1
%       F3 // transpose_local
%     MPI_Wait D_(k-1)
%     MPI_Isend D_k
%     MPI_Irecv D_k
%   MPI_Wait D_n
%
% One-sided algorithm:
% - Decompose communication
%   L
%     L
%       C1
%   L
%     L
%       F3 // transpose_local
%   MPI_Win_create
%   MPI_Win_fence
%   L
%     L
%       L
%         MPI_Put
%         MPI_Win_fence
% - Fuse loops
%   MPI_Win_create
%   MPI_Win_fence
%   L
%     L
%       C1
%       F3 // transpose_local
%       L
%         MPI_Put
%         MPI_Win_fence
% - Move wait down / out
%   MPI_Win_create
%   MPI_Win_fence
%   L
%     L
%       C1
%       L
%         F3 // transpose_local
%       MPI_Put
%     MPI_Win_fence

%% Overall workflow %%
\begin{algorithm}
{\scriptsize
\begin{algorithmic}
\Require Source code and hardware configuration
\Ensure Optimized source code

\ForAll{cco\_block with cco\_dir \textbf{in} source code}
  \ForAll{function f with decompose\_dir \textbf{in} cco\_block}
    \State decompose\_computation(f, decompose\_dir)
  \EndFor

  % Apply skeleton-based inlining to functions that are complicated
  % but won’t be modified for cco transformations
  \ForAll{function f with inline\_func\_dir \textbf{in} cco\_block}
    \State inline\_func(f)
  \EndFor

  \State skeletons = set()
  \ForAll{function f with inline\_skeleton\_dir \textbf{in} cco\_block}
    \State inlined\_code = inline\_skeleton(f)
    \State add\_to\_set(skeletons, inlined\_code)
  \EndFor

  % Reorganize the computation and communication
  % to increase the amount of computation in the communication window
  \State overlap\_comm\_comp(cco\_block, cco\_dir)

  \ForAll{sk \textbf{in} skeletons}
    \State reverse\_inline\_skeleton(sk)
  \EndFor
\EndFor

\end{algorithmic}
\caption{Overall transformation algorithm}
\label{alg:workflow}
}%\scriptsize
\end{algorithm}

%% Computation-communication overlap %%

\begin{algorithm}
{\scriptsize
\begin{algorithmic}
\Require Source code with MPI cco pragma,
\State where the computation have been already decomposed, and
\State the functions have been selectively inlined so that there are no procedure boundaries
\Ensure Code where communication is decomposed and overlapped with computation

% Overlap communication with the computation in input
% - input: the cco block to optimize
% - cco\_directive: the cco pragma describing the computation pattern
\Function{overlap\_comm\_comp}{input, cco\_dir}
  \State comm\_group = get\_comm\_group(cco\_dir) % get the computation group from the cco pragma
  \State comms = find\_matching\_comms(comm\_group, input) % find individual communication operations
  \ForAll{MPI call comm \textbf{in} comms}
    \If{has\_decompose\_dir(comm, cco\_dir)}
      \State decompose\_comm(comm, cco\_dir) % according to the pragma, split communication
    \EndIf
    \If{is\_blocking\_comm(comm)}
      \State (comm, wait) = blocking\_to\_non\_blocking\_comm(comm) % split into blocking wait and non-blocking comm
    \Else
      \State wait = find\_blocking\_wait(comm, input) % find the wait for non-blocking communication
    \EndIf

    % Synchronized wait should be moved before moving asynchronized comm
    \State move\_wait(wait, input)  % continuously move sync wait down or out to the upper block

    \State scope = get\_enclosing\_block(wait) % get the current enclosing block outside wait
    \State move\_comm(comm, scope)  % continuously move async comm up or merge with the previous loop
  \EndFor
\EndFunction

% Move the synchronized wait down or out of the enclosing block
% - wait: code block that contains synchronized wait
% - scope: the top level block that wait should not go outside of
\Function{move\_wait}{wait, scope}
  \While{\textbf{true}} % continuously move wait down or out  to upper block
    \If{has\_next\_stmt(wait, scope)}
      \State next = get\_next\_stmt(wait) % get the next statement in the enclosing block
      \If{indep(wait, next)}
        \State swap\_stmt(wait, next) % move wait after next statement
        \State \textbf{continue}
      \EndIf
    \EndIf
    \If{has\_enclosing\_block(wait, scope)}
      \State parent = get\_enclosing\_block(wait)
      \If{block\_ends\_with(block, wait) \textbf{and} indep(wait, block)}
        \State move\_stmt\_out(wait, block) % move wait out behind the block
      \EndIf
      \State \textbf{continue}
    \EndIf
    \State \textbf{break}
  \EndWhile
\EndFunction

% Move the asynchronized communication up or fuse with the previous loop
% - comm: Code block that contains asynchronized communication
% - scope: the top level block that wait should not go outside of
\Function{move\_comm}{comm, scope}
  \While{\textbf{true}} % continuously move async comm up or merge with the previous loop
    \If{has\_prev\_stmt(comm, scope)}
      \State prev = get\_prev\_stmt(comm, scope) % get the previous statement in the enclosing block
      \If{indep(comm, prev)}
        \State swap\_stmt(comm, prev):  % move comm before prev statement
        \State \textbf{continue}
      \EndIf
      \If{is\_loop(comm) \textbf{and} is\_loop(prev) \textbf{and} fusable(comm, prev)}
        \State fused\_loop = fuse\_loops(comm, prev) % If both are loops, try to fuse them
        \State comm = fused\_loop
        \State \textbf{continue}
      \EndIf
    \EndIf
    \If{has\_enclosing\_block(comm, scope)}
      \State comm = get\_enclosing\_block(comm)
      \State \textbf{continue}
    \EndIf
    \State \textbf{break}
  \EndWhile
\EndFunction

\end{algorithmic}
\caption{CCO transformation algorithm}
\label{alg:cco}
}%\scriptsize
\end{algorithm}

% EOF
