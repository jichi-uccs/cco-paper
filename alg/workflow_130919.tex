% workflow.tex
% 9/5/2013
% The hotspot and hotpath algorithm figure.
%
% Algorithms:
% - region profiling
% - hot spot prediction
% - hot path extraction
%
%
% Example:
% - Input: FT
%   F1 // cffts
%     L
%       L
%        C1 // cfftz
%   F2 // transpose
%     F3 // transpose_local
%       L
%         L
%           C2
%     F4 // transpose_global
%       MPI_Alltoall
%
% - Inline & decompose computation:
%   L
%     L
%       C1
%   L
%     L
%       F3 // transpose_local
%   MPI_Alltoall
%
% - Output: one-sided:
%    MPI_Win_create
%    MPI_Win_fence
%    L
%      L
%        C1
%        F3
%        L
%          MPI_Put
%    MPI_Win_fence
%
% - Output: non-blocking, see: http://www.2decomp.org/occ.html
%   L
%     C1
%     F3 // transpose_local
%   MPI_Isend D_1
%   MPI_Irecv D_1
%   L k=2,n
%     L
%       C1
%       F3 // transpose_local
%     MPI_Wait D_(k-1)
%     MPI_Isend D_k
%     MPI_Irecv D_k
%   MPI_Wait D_n
%
% One-sided algorithm:
% - Decompose communication
%   L
%     L
%       C1
%   L
%     L
%       F3
%   MPI_Win_create
%   MPI_Win_fence
%   L
%     L
%       L
%         MPI_Put
%         MPI_Win_fence
% - Fuse loops
%   MPI_Win_create
%   MPI_Win_fence
%   L
%     L
%       C1
%       F3
%       L
%         MPI_Put
%         MPI_Win_fence
% - Move wait down / out
%   MPI_Win_create
%   MPI_Win_fence
%   L
%     L
%       C1
%       F3
%       L
%         MPI_Put
%     MPI_Win_fence
% - Move communication up, nothing happens

%% Overall workflow %%
\begin{algorithm}
{\scriptsize
\begin{algorithmic}
\Require Source code and hardware configuration
\Ensure Optimized source code

\ForAll{cco\_block with cco\_pragma \textbf{in} source code}
  \ForAll{function f with decompose\_pragma \textbf{in} cco\_block}
    \State decompose\_computation(f, decompose\_pragma)
  \EndFor

  % Apply skeleton-based inlining to functions that are complicated
  % but won’t be modified for cco transformations
  \ForAll{function f with inline\_func\_pragma \textbf{in} cco\_block}
    \State inline\_func(f)
  \EndFor

  \State skeletons = set()
  \ForAll{function f with inline\_skeleton\_pragma \textbf{in} cco\_block}
    \State inlined\_code = inline\_skeleton(f)
    \State skeletons.add(inlined\_code)
  \EndFor

  % Reorganize the computation and communication
  % to increase the amount of computation in the communication window
  \State overlap\_comm\_comp(cco\_block, cco\_pragma)

  \ForAll{sk \textbf{in} skeletons}
    \State reverse\_inline\_skeleton(sk)
  \EndFor
\EndFor

\end{algorithmic}
\caption{Overall transformation algorithm}
\label{alg:workflow}
}%\scriptsize
\end{algorithm}

%% Computation-communication overlap %%

\begin{algorithm}
{\scriptsize
\begin{algorithmic}
\Require Source code with MPI cco pragma,
\State where the computation have been already decomposed, and
\State the functions have been selectively inlined so that there are no procedure boundaries
\Ensure Code where communication is decomposed and overlapped with computation

% Overlap communication with the computation in input
% - input: the cco block to optimize
% - cco\_directive: the cco pragma describing the computation pattern
\Function{overlap\_comm\_comp}{input, cco\_pragma}
  \State comm\_group = get\_comm\_group(cco\_pragma) % get the computation group from the cco pragma
  \State comms = find\_matching\_comms(comm\_group, input) % find individual communication operations
  \State indep\_stmts = find\_indep\_stmts(comm\_group, input) % find individual communication operations
  \ForAll{MPI comm \textbf{in} comms}
    \If{has\_decompose\_pragma(comm, cco\_pragma)}
      \State comm = decompose\_communication(comm, cco\_pragma)
    \EndIf
    \State (decl, comm, wait) = blocking\_to\_nonblocking(comm) % split into blocking wait and non-blocking comm
    \State insert\_decl(decl, input)
    \State (before, in1, in2, after) = split\_stmt\_block(input, comm, wait);
    \State move\_comm\_up(in1, indep\_stmts, before);
    \State move\_comm\_down(in2, indep\_stmts, after);
  \EndFor
\EndFunction

\Function{move\_comm\_up}{comm, indep\_stmts, before}
  \ForAll{stmt in before}
    \If{stmt in indep\_stmts}
      \State swap\_stmts(comm, stmt)
    \Else
      \If{can\_fuse\_or\_merge(comm, stmt)}
        \State block = merge\_or\_fuse(comm, stmt)
        \State comm = get\_last\_stmt(block)
        \State move\_comm\_up(comm, indep\_stmts, block)
      \EndIf
      \State \textbf{break}
    \EndIf
  \EndFor
\EndFunction

\Function{move\_comm\_down}{wait, indep\_stmts, after}
  \ForAll{stmt in after}
    \If{stmt in indep\_stmts}
      \State swap\_stmts(wait, stmt)
    \Else
      \If{can\_fuse\_or\_merge(wait, stmt)}
        \State block = merge\_or\_fuse(wait, stmt)
        \State wait = get\_first\_stmt(block)
        \State move\_comm\_down(wait, indep\_stmts, block)
      \EndIf
      \State \textbf{break}
    \EndIf
  \EndFor
\EndFunction

\end{algorithmic}
\caption{CCO transformation algorithm}
\label{alg:cco}
}%\scriptsize
\end{algorithm}

% EOF

%\Function{icpp\_up}{comm}
%  \State block = get\_enclosing\_stmt(comm)
%  \State indep\_stmts = find\_indep\_stmts(comm, block)
%  \State (before, in, after) = split\_stmt\_block(block, comm)
%  \State move\_comm\_up(in, indep\_stmts, before)
%\EndFunction
%
%\Function{icpp\_down}{wait}
%  \State block = get\_enclosing\_stmt(wait)
%  \State indep\_stmts = find\_indep\_stmts(wait, block)
%  \State (before, in, after) = split\_stmt\_block(block, wait)
%  \State move\_comm\_down(in, indep\_stmts, after)
%\EndFunction
%
%\Function{move\_comm}{comm, wait, scope}
%  \State icpp\_up(comm)
%  \If{has\_prev\_stmt(comm, scope)}
%    \State prev = get\_prev\_stmt(comm)
%    \If{indep(prev, comm) \textbf{and} can\_merge\_or\_fuse(prev, comm)}
%      \State comm = merge\_blocks\_or\_fuse\_loops(prev, comm)
%      \State move\_comm(comm, wait, scope)
%    \EndIf
%  \EndIf
%
%  \State icpp\_down(wait)
%  \If{can\_move\_wait\_out(wait)}
%    \State comm = move\_stmt\_out(wait)
%    \State move\_comm(comm, wait, scope)
%  \EndIf
%\EndFunction
%
%\Function{move\_stmt\_out}{stmt}
%  \State block = get\_enclosing\_stmt(stmt)
%  \State insert\_stmt\_after(stmt, block)
%  \State remove\_stmt(stmt, block)
%  \State \Return block
%\EndFunction

%\Function{get\_same\_level}{x, y, scope}
%  \If{in\_the\_same\_block(x, y)}
%    \State \Return (True, x, y)
%  \EndIf
%  \If{has\_enclosing\_block(x, scope)}
%    \State parent = get\_enclosing\_block(x)
%    \State (ok, x1, y1) = get\_same\_level(parent, y, scope)
%    \If{ok}
%      \State \Return (True, x1, y1)
%    \EndIf
%  \EndIf
%  \If{has\_enclosing\_block(y, scope)}
%    \State parent = get\_enclosing\_block(y)
%    \State (ok, x1, y1) = get\_same\_level(x, parent, scope)
%    \If{ok}
%      \State \Return (True, x1, y1)
%    \EndIf
%  \EndIf
%  \State \Return (False, None, None)
%\EndFunction

% Increase the body size of block or its parent block
% - block: the block
% - scope: the top level block that wait should not go outside of
%\Function{reorganize\_block}{cur, scope}
%  \If{has\_enclosing\_block(cur, scope)}
%    \State parent = get\_enclosing\_block(cur)
%    \If{has\_prev\_stmt(cur)}
%      \State prev = get\_prev\_stmt(cur)
%      \If{indep(cur, prev)}
%        \If{is\_block(cur) and is\_block(prev)}
%          \State cur = merge\_block(cur, prev)
%          \State reorganize\_block(cur, scope)
%        \ElsIf{is\_loop(cur) and is\_loop(prev) and fusable(cur, prev)}
%          \State cur = merge\_block(cur, prev)
%          \State reorganize\_block(cur, scope)
%        %\Else
%        % TODO: What to do if prev is not a loop?!
%        \EndIf
%      \EndIf
%      \State next = get\_next\_stmt(cur)
%      \If{indep(cur, next)}
%        \If{is\_block(cur) and is\_block(next)}
%          \State cur = merge\_block(cur, next)
%          \State reorganize\_block(cur, scope)
%        \ElsIf{is\_loop(cur) and is\_loop(next) and fusable(cur, next)}
%          \State cur = merge\_block(cur, next)
%          \State reorganize\_block(cur, scope)
%        %\Else
%        % TODO: What to do if next is not a loop?!
%        \EndIf
%      \EndIf
%    \EndIf
%  \EndIf
%\EndFunction

% Move the synchronized wait down or out of the enclosing block
% - wait: code block that contains synchronized wait
% - scope: the top level block that wait should not go outside of
%\Function{move\_wait}{wait, scope}
%  \If{has\_next\_stmt(wait, scope)}
%    \State next = get\_next\_stmt(wait) % get the next statement in the enclosing block
%    \If{indep(wait, next)}
%      \State swap\_stmt(wait, next) % move wait after next statement
%      \State move\_wait(wait, scope)
%    \EndIf
%  \EndIf
%  \If{has\_enclosing\_block(wait, scope)}
%    \State parent = get\_enclosing\_block(wait)
%    \If{block\_ends\_with(block, wait) \textbf{and} indep(wait, block)}
%      \State move\_stmt\_out(wait, block) % move wait out behind the block
%      \State move\_wait(wait, scope)
%    \EndIf
%  \EndIf
%\EndFunction

% Move the asynchronized communication up or fuse with the previous loop
% - comm: Code block that contains asynchronized communication
% - scope: the top level block that wait should not go outside of
%\Function{move\_comm}{comm, scope}
%  \If{has\_prev\_stmt(comm, scope)}
%    \State prev = get\_prev\_stmt(comm, scope) % get the previous statement in the enclosing block
%    \If{indep(comm, prev)}
%      \State swap\_stmt(comm, prev):  % move comm before prev statement
%      \State move\_comm(comm, scope)
%    \EndIf
%  \EndIf
%  \If{has\_enclosing\_block(comm, scope)}
%    \State parent = get\_enclosing\_block(wait)
%    \State move\_comm(parent, scope)
%  \EndIf
%\EndFunction
